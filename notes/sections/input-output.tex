\section{Input and Output}
\begin{frame}
    \frametitle{Input and Output}
    \begin{itemize}
        \item Input and output are essential for interacting with the user.
        \item Python provides built-in functions for reading input and displaying output.
        \item The \texttt{input()} function reads input from the user.
        \item The \texttt{print()} function displays output to the user.
    \end{itemize}
\end{frame} 

\begin{frame}[fragile]{The \texttt{input()} Function}
    \begin{itemize}
        \item The \texttt{input()} function reads a line of text from the user.
        \item The text is returned as a string.
        \item The user must press the \texttt{Enter} key to submit the input.
    \end{itemize}
    \begin{lstlisting}[style=colorful, language=Python]
>>> name = input("Enter your name: ")
Enter your name: David
>>> print(name)
David
    \end{lstlisting}    
\end{frame}

\begin{frame}[fragile]{}
    \begin{lstlisting}[style=colorful, language=Python]
>>> print("Hello")
Hello
>>> print("Hello", "World")
Hello World
>>> print("Hello", "World", sep="***")
Hello***World
>>> print("Hello", "World", end="***")
Hello World***
    \end{lstlisting}
\end{frame}

\begin{frame}[fragile]
    \frametitle{Formatting Strings}
    \begin{lstlisting}[style=colorful, language=Python]
>>> name = "David"
>>> age = 32
>>> print(f"My name is {name}. I am {age} years old.")
My name is David. I am 32 years old.    
    \end{lstlisting}  
\end{frame}


\begin{frame}{String Formatting Modifiers in Python (Part 1)}
    \begin{table}[]
        \centering
        \begin{tabular}{|l|l|p{6cm}|}
            \hline
            \textbf{Modifier} & \textbf{Example} & \textbf{Description} \\ \hline
            \texttt{number} & \texttt{\%20d} & Put the value in a field width of 20 \\ \hline
            \texttt{-} & \texttt{\%<20d} & Put the value in a field 20 characters wide, left-justified \\ \hline
            \texttt{+} & \texttt{\%>20d} & Put the value in a field 20 characters wide, right-justified \\ \hline
            \texttt{0} & \texttt{\%020d} & Put the value in a field 20 characters wide, fill in with leading zeros \\ \hline
            \texttt{.} & \texttt{\%20.2f} & Put the value in a field 20 characters wide with 2 characters to the right of the decimal point \\ \hline
        \end{tabular}
        \caption{String Formatting Modifiers in Python (Part 1)}
    \end{table}
\end{frame}

\begin{frame}[fragile]{String Formatting Modifiers in Python (Part 2)}
    \begin{lstlisting}[style=colorful, language=Python]
# In mac >20.2f is left justified and <20.2f is right justified
>>> print(f"The price of the product is {1000.4567:>20.2f}") 
The price of the product is           1,000.46
    \end{lstlisting}
\end{frame}
