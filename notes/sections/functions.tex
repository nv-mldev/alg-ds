\section{functions}
\label{sec:functions}  


\begin{frame}{Python Functions}
    \begin{itemize}
        \item \textbf{Reusable Code Blocks:} Functions encapsulate blocks of code for repeated use.
        \item \textbf{Modularity:} They break down large programs into smaller, manageable parts.
        \item \textbf{Abstraction:} Functions hide implementation details, focusing on what they do.
        \item \textbf{Inputs and Outputs:}
            \begin{itemize}
                \item Functions can accept \textbf{parameters} (inputs).
                \item They can return \textbf{values} (outputs) using the \texttt{return} statement.
            \end{itemize}
        \item \textbf{Definition:}
            \begin{itemize}
                \item Defined using the \texttt{def} keyword.
                \item Indented code block follows the function definition.
            \end{itemize}
        \item \textbf{Benefits:}
            \begin{itemize}
                \item Code reusability.
                \item Improved readability.
                \item Better organization.
                \item Easier debugging.
            \end{itemize}
    \end{itemize}
    
    \end{frame}



\begin{frame}
    \frametitle{Functions}
\begin{itemize}
    \item \textbf{Function} is a block of code that performs a specific task.
    \item \textbf{Function Declaration} is a statement that defines a function.
    \item \textbf{Function Call} is a statement that executes a function.
    \item \textbf{Function Definition} is a statement that defines the code that the function executes.
    \item \textbf{Function Prototype} is a statement that declares the function's name, return type, and parameters.
    \item \textbf{Function Parameters} are variables that are passed to a function.
\end{itemize}
\end{frame}
    
\begin{frame}
    \frametitle{Functions}
    \begin{itemize}
    \item \textbf{Function Arguments} are values that are passed to a function.
    \item \textbf{Return Type} is the data type of the value that a function returns.
    \item \textbf{Return Statement} is a statement that returns a value from a function.
    \item \textbf{Void Function} is a function that does not return a value.
    \item \textbf{Value-Returning Function} is a function that returns a value.
    \item \textbf{Function Overloading} is the ability to define multiple functions with the same name but different parameters.
    \item \textbf{Function Recursion} is the ability of a function to call itself.
    \end{itemize}
\end{frame}

\begin{frame}[fragile]
    \frametitle{Function Definition- Square root Funtion }
    \begin{lstlisting}[style=colorful, language=Python]
def square_root(x):
'''
Function to calculate the square root of a number using Newton's method
'''
    guess = x / 2
    while abs(guess**2 - x) > 0.0001:
        guess = (guess + (x / guess)) / 2
    return guess
    \end{lstlisting}
\end{frame}