\section{Exception Handling}
\begin{frame}
    \frametitle{Exception Handling}
    \begin{itemize}
        \item Exception handling allows you to gracefully handle errors in your program.
        \item Exceptions are raised when an error occurs during program execution.
        \item You can catch and handle exceptions using \texttt{try} and \texttt{except} blocks.
    \end{itemize}
\end{frame} 

\begin{frame}
    \frametitle{Type of errors in Python}
    \begin{itemize}
        \item Syntax errors
        \item Runtime errors
        \item Logical errors
    \end{itemize}   
\end{frame} 

\begin{frame}[fragile]{Syntax Errors}
    \begin{itemize}
        \item Syntax errors occur when the code is not written correctly.
        \item The Python interpreter cannot execute code with syntax errors.
        \item Syntax errors are detected during the parsing phase.
    \end{itemize}
    \begin{lstlisting}[style=colorful, language=Python]
# Syntax error: missing colon
if x > 5
    print("x is greater than 5")
    \end{lstlisting}
\end{frame} 

\begin{frame}[fragile]{Runtime Errors}
    \begin{itemize}
        \item Runtime errors occur during program execution.
        \item They are also known as exceptions.
        \item Runtime errors can be handled using exception handling.
    \end{itemize}
    \begin{lstlisting}[style=colorful, language=Python]
# Runtime error: division by zero
x = 10 / 0
    \end{lstlisting}
\end{frame} 

\begin{frame}[fragile]{Logical Errors}
    \begin{itemize}
        \item Logical errors occur when the program does not produce the expected output.
        \item The code is syntactically correct and does not raise any exceptions.
        \item Logical errors are the most challenging to debug.
    \end{itemize}
    \begin{lstlisting}[style=colorful, language=Python]
# Logical error: incorrect calculation
x = 10
y = 20
result = x + y  # Should be x * y
    \end{lstlisting}
\end{frame} 

\begin{frame}[fragile]{Exception Handling Syntax}
    \begin{lstlisting}[style=colorful, language=Python]
try:            # Try block
    # code that may raise an exception
except Exception as e:  # Exception block
    # code to handle the exception
    \end{lstlisting}
    \begin{itemize}
        \item The \texttt{try} block contains the code that may raise an exception.
        \item The \texttt{except} block contains the code to handle the exception.
        \item The \texttt{except} block is executed only if an exception occurs in the \texttt{try} block.
    \end{itemize}
\end{frame}

\begin{frame}[fragile]{Exception Handling Example}
    \begin{lstlisting}[style=colorful, language=Python]
try:
    x = 10 / 0  # Division by zero
except ZeroDivisionError as e:
    print("Error:", e)
    \end{lstlisting}
    \begin{itemize}
        \item In this example, the code inside the \texttt{try} block raises a \texttt{ZeroDivisionError}.
        \item The \texttt{except} block catches the exception and prints an error message.
    \end{itemize}
\end{frame} 

\begin{frame}[fragile]{Exception Handling Example (Part 2)}
    \begin{lstlisting}[style=colorful, language=Python]
>>> a_number = int(input("Please enter an integer "))
Please enter an integer -23
>>> print(math.sqrt(a_number))
Traceback (most recent call last):
File "<pyshell#102>", line 1, in <module>
print(math.sqrt(a_number))
ValueError: math domain error
>>> try:
print(math.sqrt(a_number))
except:
print("Bad Value for square root")
print("Using absolute value instead")
print(math.sqrt(abs(a_number)))
Bad Value for square root
Using absolute value instead
4.795831523312719
>>>
 \end{lstlisting}
\end{frame}




\begin{frame}[fragile]{Multiple Exceptions}
    \begin{lstlisting}[style=colorful, language=Python]
try:
    x = 10 / 0  # Division by zero
except ZeroDivisionError as e:
    print("Error:", e)
except Exception as e:
    print("An error occurred:", e)
    \end{lstlisting}
    \begin{itemize}
        \item You can catch multiple exceptions by adding multiple \texttt{except} blocks.
        \item The \texttt{except} block that matches the exception type is executed.
    \end{itemize}
\end{frame} 

\begin{frame}[fragile]{Finally Block}
    \begin{lstlisting}[style=colorful, language=Python]
try:
    x = 10 / 0  # Division by zero
except ZeroDivisionError as e:
    print("Error:", e)
finally:
    print("Finally block")
    \end{lstlisting}
    \begin{itemize}
        \item The \texttt{finally} block is executed regardless of whether an exception occurs.
        \item The \texttt{finally} block is useful for releasing resources or cleaning up.
    \end{itemize}
\end{frame}     


\begin{frame}[fragile]{Example of Finally Block}
    \begin{lstlisting}[style=colorful, language=Python]
def read_file(file_path):
    try:
        file = open(file_path, 'r')
        content = file.read()
    except FileNotFoundError as e:
        print(f"Error: {e}")
    else:
        print(f"File content:\n{content}")
    finally:
        # this is to handle the case when the file variable is not initialized
        try:
            file.close()
        except NameError:
            print("File is not open")
        print("Execution of the try-except block is complete.")
    \end{lstlisting}
\end{frame}

\begin{frame}[fragile]{Raise an Exception Example}
    \begin{lstlisting}[style=colorful, language=Python]
if a_number < 0:
    raise RuntimeError("You can't use a negative number")
else:
    print(math.sqrt(a_number))
>>> math.sqrt(-23)
Traceback (most recent call last):
File "<pyshell#20>", line 2, in <module>
raise RuntimeError("You can't use a negative number")
RuntimeError: You can't use a negative number

    \end{lstlisting}
    \begin{itemize}
        \item You can raise an exception using the \texttt{raise} statement.
        \item The \texttt{raise} statement raises an exception with a specified message.
    \end{itemize}
\end{frame} 

\begin{frame}{Common Exception Types in Python (Part 1)}
    \begin{itemize}
        \item \textbf{BaseException}: The base class for all built-in exceptions.
        \item \textbf{Exception}: The base class for all non-exit exceptions.
        \item \textbf{ArithmeticError}: The base class for all errors that occur for numeric calculations.
            \begin{itemize}
                \item \textbf{OverflowError}: Raised when the result of an arithmetic operation is too large to be expressed.
                \item \textbf{ZeroDivisionError}: Raised when division or modulo by zero takes place.
                \item \textbf{FloatingPointError}: Raised when a floating point operation fails.
            \end{itemize}
        \item \textbf{AttributeError}: Raised when an attribute reference or assignment fails.
        \item \textbf{EOFError}: Raised when the \texttt{input()} function hits an end-of-file condition.
    \end{itemize}
\end{frame}

\begin{frame}{Common Exception Types in Python (Part 2)}
    \begin{itemize}
        \item \textbf{ImportError}: Raised when an import statement fails to find the module definition or when a \texttt{from ... import} fails to find a name that is to be imported.
            \begin{itemize}
                \item \textbf{ModuleNotFoundError}: A subclass of \textbf{ImportError} raised when a module could not be located.
            \end{itemize}
        \item \textbf{IndexError}: Raised when a sequence subscript is out of range.
        \item \textbf{KeyError}: Raised when a dictionary key is not found.
        \item \textbf{KeyboardInterrupt}: Raised when the user hits the interrupt key (normally \texttt{Control-C} or \texttt{Delete}).
        \item \textbf{MemoryError}: Raised when an operation runs out of memory.
        \item \textbf{NameError}: Raised when a local or global name is not found.
            \begin{itemize}
                \item \textbf{UnboundLocalError}: A subclass of \textbf{NameError} raised when a local variable is referenced before it has been assigned.
            \end{itemize}
    \end{itemize}
\end{frame}

\begin{frame}{Common Exception Types in Python (Part 3)}
    \begin{itemize}
        \item \textbf{OSError}: Raised when a system function returns a system-related error.
            \begin{itemize}
                \item \textbf{FileNotFoundError}: Raised when a file or directory is requested but doesn't exist.
                \item \textbf{PermissionError}: Raised when trying to run an operation without the adequate access rights.
                \item \textbf{TimeoutError}: Raised when a system function timed out at the system level.
            \end{itemize}
        \item \textbf{RuntimeError}: Raised when an error is detected that doesn't fall in any of the other categories.
            \begin{itemize}
                \item \textbf{NotImplementedError}: Raised when an abstract method that needs to be implemented in an inherited class is not actually implemented.
            \end{itemize}
        \item \textbf{SyntaxError}: Raised when the parser encounters a syntax error.
            \begin{itemize}
                \item \textbf{IndentationError}: Raised when there is an incorrect indentation.
                    \begin{itemize}
                        \item \textbf{TabError}: Raised when indentation consists of inconsistent tabs and spaces.
                    \end{itemize}
            \end{itemize}
    \end{itemize}
\end{frame}

\begin{frame}{Common Exception Types in Python (Part 4)}
    \begin{itemize}
        \item \textbf{TypeError}: Raised when an operation or function is applied to an object of inappropriate type.
        \item \textbf{ValueError}: Raised when an operation or function receives an argument that has the right type but an inappropriate value.
            \begin{itemize}
                \item \textbf{UnicodeError}: Raised when a Unicode-related encoding or decoding error occurs.
                    \begin{itemize}
                        \item \textbf{UnicodeEncodeError}: Raised when a Unicode-related error occurs during encoding.
                        \item \textbf{UnicodeDecodeError}: Raised when a Unicode-related error occurs during decoding.
                        \item \textbf{UnicodeTranslateError}: Raised when a Unicode-related error occurs during translation.
                    \end{itemize}
            \end{itemize}
        \item \textbf{StopIteration}: Raised by the \texttt{next()} function to indicate that there are no further items produced by the iterator.
        \item \textbf{AssertionError}: Raised when an \texttt{assert} statement fails.
    \end{itemize}
\end{frame}