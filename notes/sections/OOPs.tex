\section{Obejct Oriented Programming}
\label{sec:OOPs}

\begin{frame}
    \frametitle{Object Oriented Programming}
    \begin{itemize}
        \item \textbf{Object:} An object is a real-world entity that has a state and behavior.
        \item \textbf{Class:} A class is a blueprint for creating objects.
        \item \textbf{Encapsulation:} Encapsulation is the bundling of data and methods that operate on the data.
        \item \textbf{Abstraction:} Abstraction is the process of hiding the implementation details and showing only the functionality.
        \item \textbf{Inheritance:} Inheritance is the mechanism of creating a new class from an existing class.
        \item \textbf{Polymorphism:} Polymorphism is the ability of an object to take on many forms.
    \end{itemize}
\end{frame}

\begin{frame}
    \frametitle{class syntax}
    \begin{itemize}
        \item A class is defined using the \texttt{class} keyword.
        \item The class definition consists of a class name and a code block.
        \item The code block contains class attributes and methods.
        \item The class attributes are variables that store data.
        \item The class methods are
            \begin{itemize}
                \item Functions that perform operations on the data.
                \item Defined using the \texttt{def} keyword.
                \item The first parameter of a method is \texttt{self}.
            \end{itemize}
    \end{itemize}       
\end{frame}

\begin{frame}[fragile]
    \frametitle{Creating a Class}
    \begin{lstlisting}[style=colorful, language=Python]
class Robot:
    def __init__(self, name, identifier):
        self.name = name
        self.identifier = identifier
    def say_hello(self):
        print(f"Hello. My name is {self.name}. My ID is {self.identifier}.")
    \end{lstlisting}
\end{frame} 

\begin{frame}
\frametitle{Defaulf Constructor}
\begin{itemize}
    \item The \texttt{\_\_init\_\_()} method is a special method called a constructor.
    \item It is called when an object is created.
    \item The \texttt{self} parameter refers to the object itself.
    \item The \texttt{self} parameter is used to access the object's attributes and methods.
    \item The \texttt{self} parameter is passed implicitly when calling a method.
\end{itemize}       
\end{frame}

\begin{frame}
    \frametitle{Default methods}
    \begin{itemize}
        \item The \texttt{\_\_str\_\_()} method is a special method that returns a string representation of an object.
        \item The \texttt{\_\_str\_\_()} method is called when an object is printed.
        \item The \texttt{\_\_repr\_\_()} method is a special method that returns a string representation of an object.
        \item The \texttt{\_\_repr\_\_()} method is called when an object is printed in the interactive shell.
    \end{itemize}   
\end{frame}

\begin{frame}
\frametitle{Example of Default methods}
\begin{lstlisting}[style=colorful, language=Python]
class Robot:
    def __init__(self, name, identifier):
        self.name = name
        self.identifier = identifier
    def say_hello(self):
        print(f"Hello. My name is {self.name}. My ID is {self.identifier}.")
    def __str__(self):
        return f"Robot({self.name}, {self.identifier})"
    def __repr__(self):
        return f"Robot({self.name}, {self.identifier})"
    
\end{lstlisting}
\end{frame}
