\section{Basic data structures}
\label{sec:basic-data-structures}

\begin{frame}
    \frametitle{linear data structures}
    \begin{itemize}
        \item \textbf{Arrays:} A collection of elements stored in contiguous memory locations.
        \item \textbf{Linked Lists:} A collection of elements stored in non-contiguous memory locations.
        \item \textbf{Stacks:} A collection of elements with last-in, first-out (LIFO) access.
        \item \textbf{Queues:} A collection of elements with first-in, first-out (FIFO) access.
    \end{itemize}
\end{frame} 

\begin{frame}
    \frametitle{Why called linear data structures?}
    \begin{itemize}
        \item \textbf{Sequential Access:} Elements are accessed in a sequential manner.
        \item \textbf{Single Level:} They have a single level of elements, forming a linear sequence.
        \item \textbf{Memory Allocation:} Memory is allocated in a linear fashion.
    \end{itemize}
\end{frame}
\subsection{Stack}
\begin{frame}
\frametitle{What is a Stack}
\begin{itemize}
    \item \textbf{Stack} is a linear data structure that follows the Last-In, First-Out (LIFO) principle.
    \item \textbf{Operations:}
        \begin{itemize}
            \item \textbf{Push:} Adds an element to the top of the stack.
            \item \textbf{Pop:} Removes the top element from the stack.
            \item \textbf{Peek:} Returns the top element without removing it.
        \end{itemize}
    \item \textbf{Applications:}
        \begin{itemize}
            \item Function calls.
            \item Expression evaluation.
            \item Undo mechanisms.
        \end{itemize}
\end{frame}

