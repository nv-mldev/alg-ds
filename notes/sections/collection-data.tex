\section{Collection Data Types}
\begin{frame}
    \frametitle{Builtin Collection Data Types}
    \begin{itemize}
        \item Python provides several collection data types
        \item These types are used to store multiple values in a single variable
        \item Common collection data types include:
        \begin{itemize}
            \item Lists
            \item Tuples
            \item Sets
            \item Dictionaries
        \end{itemize}
    \end{itemize}   
\end{frame}

\begin{frame}
    \frametitle{Lists}
    \begin{itemize}
        \item Lists are ordered collections of items
        \item Lists can contain items of different types
        \item Lists are mutable, meaning they can be changed after creation
        \item Lists are created using square brackets \texttt{[]}
    \end{itemize}
\end{frame}

\begin{frame}[fragile]{Creating Lists}
    \begin{lstlisting}[style=colorful, language=Python]
# Create an empty list
my_list = []
# Create a list with items
my_list = [1, 2, 3, 4, 5]  
my_another_list = ['a', 'b', 'c', False, 3, 0.11]  
    \end{lstlisting}
\end{frame}

\begin{frame}[fragile]{Accessing List Items}
    \begin{lstlisting}[style=colorful, language=Python]
# Access the first item in the list
print(my_list[0])
# Access the last item in the list
print(my_list[-1])
    \end{lstlisting}
\end{frame} 

\begin{frame}[fragile]{Slicing Lists}
    \begin{lstlisting}[style=colorful, language=Python]
# Get the first three items in the list
print(my_list[:3])
# Get the last three items in the list
print(my_list[-3:])
    \end{lstlisting}
\end{frame}             


\begin{frame}[fragile]{Adding Items to a List}
    \begin{lstlisting}[style=colorful, language=Python]
# Append an item to the end of the list
my_list.append(6)
# Insert an item at a specific index
my_list.insert(0, 0)
    \end{lstlisting}
\end{frame}     


\begin{frame}[fragile]{List Operations}
    \begin{lstlisting}[style=colorful, language=Python]
# Concatenate two lists
new_list = my_list + [7, 8, 9]
# Repeat a list
repeated_list = my_list * 2
    \end{lstlisting}
\end{frame}

\begin{frame}[fragile]{Mutability of Lists in Python}
    \begin{itemize}
        \item Lists in Python are mutable, meaning their contents can be changed after creation.
        \item Example: Modifying a list through a reference.
    \end{itemize}
    \begin{lstlisting}[style=colorful, language=Python]
# Create a list
a = [1]
# Assign the list to another variable
b = a
# Modify the list through the original variable
a[0] = 5
# Print the modified list through the reference
print(b)  # Output: [5]
    \end{lstlisting}
    \begin{itemize}
        \item Both \texttt{a} and \texttt{b} refer to the same list object.
        \item Modifying the list through \texttt{a} also affects \texttt{b}.
    \end{itemize}
\end{frame}

\begin{frame}[fragile]{List References in Python}
    \begin{lstlisting}[style=colorful, language=Python]
        # Create a list
        my_list = [1, 2, 3, 4]
        # Create a list of references to the original list
        A = [my_list] * 3
        print(A)  # Output: [[1, 2, 3, 4], [1, 2, 3, 4], [1, 2, 3, 4]]
        # Modify the original list
        my_list[2] = 45
        print(A)  # Output: [[1, 2, 45, 4], [1, 2, 45, 4], [1, 2, 45, 4]]
    \end{lstlisting}
\end{frame}

\begin{frame}{List Methods in Python (Part 1)}
    \begin{table}[]
        \centering
        \begin{tabular}{|l|l|p{4cm}|}
            \hline
            \textbf{Method Name} & \textbf{Use} & \textbf{Explanation} \\ \hline
            \texttt{append} & \texttt{a\_list.append(item)} & Adds a new item to the end of a list \\ \hline
            \texttt{insert} & \texttt{a\_list.insert(i, item)} & Inserts an item at the \texttt{i}th position in a list \\ \hline
            \texttt{pop} & \texttt{a\_list.pop()} & Removes and returns the last item in a list \\ \hline
            \texttt{pop} & \texttt{a\_list.pop(i)} & Removes and returns the \texttt{i}th item in a list \\ \hline
            \texttt{sort} & \texttt{a\_list.sort()} & Modifies a list to be sorted \\ \hline
        \end{tabular}
        \caption{List Methods in Python (Part 1)}
    \end{table}
\end{frame}

\begin{frame}{List Methods in Python (Part 2)}
    \begin{table}[]
        \centering
        \begin{tabular}{|l|l|p{4cm}|}
            \hline
            \textbf{Method Name} & \textbf{Use} & \textbf{Explanation} \\ \hline
            \texttt{reverse} & \texttt{a\_list.reverse()} & Modifies a list to be in reverse order \\ \hline
            \texttt{del} & \texttt{del a\_list[i]} & Deletes the item in the \texttt{i}th position \\ \hline
            \texttt{index} & \texttt{a\_list.index(item)} & Returns the index of the first occurrence of item \\ \hline
            \texttt{count} & \texttt{a\_list.count(item)} & Returns the number of occurrences of item \\ \hline
            \texttt{remove} & \texttt{a\_list.remove(item)} & Removes the first occurrence of item \\ \hline
        \end{tabular}
        \caption{List Methods in Python (Part 2)}
    \end{table}
\end{frame}

\begin{frame}[fragile]{The \texttt{range} Function in Python}
    \begin{itemize}
        \item The \texttt{range} function produces a range object that represents a sequence of values.
        \item Using the \texttt{list} function, you can convert the range object to a list to see its values.
    \end{itemize}
    \begin{lstlisting}[style=colorful, language=Python]
>>> range(10)
range(0, 10)
>>> list(range(10))
[0, 1, 2, 3, 4, 5, 6, 7, 8, 9]
>>> range(5, 10)
range(5, 10)
>>> list(range(5, 10))
[5, 6, 7, 8, 9]
>>> list(range(5, 10, 2))
[5, 7, 9]
>>> list(range(10, 1, -1))
[10, 9, 8, 7, 6, 5, 4, 3, 2]
    \end{lstlisting}
\end{frame}

\begin{frame}{Strings}
    \begin{itemize}
        \item Strings are sequences of characters
        \item Strings are immutable, meaning they cannot be changed after creation
        \item Strings are created using single quotes \texttt{'} or double quotes \texttt{''}
    \end{itemize}
\end{frame}     

\begin{frame}{String Methods in Python (Part 1)}
    \begin{table}[]
        \centering
        \begin{tabular}{|l|l|p{5cm}|}
            \hline
            \textbf{Method} & \textbf{Use} & \textbf{Explanation} \\ \hline
            \texttt{center} & \texttt{a\_string.center(w)} & Returns a string centered in a field of size \texttt{w} \\ \hline
            \texttt{count} & \texttt{a\_string.count(item)} & Returns the number of occurrences of \texttt{item} in the string \\ \hline
            \texttt{ljust} & \texttt{a\_string.ljust(w)} & Returns a string left-justified in a field of size \texttt{w} \\ \hline
            \texttt{lower} & \texttt{a\_string.lower()} & Returns a string in all lowercase \\ \hline
        \end{tabular}
        \caption{String Methods in Python (Part 1)}
    \end{table}
\end{frame}

\begin{frame}{String Methods in Python (Part 2)}
    \begin{table}[]
        \centering
        \begin{tabular}{|l|l|p{5cm}|}
            \hline
            \textbf{Method} & \textbf{Use} & \textbf{Explanation} \\ \hline
            \texttt{rjust} & \texttt{a\_string.rjust(w)} & Returns a string right-justified in a field of size \texttt{w} \\ \hline
            \texttt{find} & \texttt{a\_string.find(item)} & Returns the index of the first occurrence of \texttt{item} \\ \hline
            \texttt{split} & \texttt{a\_string.split(s\_char)} & Splits a string into substrings at \texttt{s\_char} \\ \hline
        \end{tabular}
        \caption{String Methods in Python (Part 2)}
    \end{table}
\end{frame}

\begin{frame}[fragile]{Sting methods}
    \begin{lstlisting}[style=colorful, language=Python]
 >>> "David"
'David'
>>> my_name = "David"
>>> my_name[3]
'i'
>>> my_name*2
'DavidDavid'
>>> len(my_name)
5
>>> my_name.upper()
'DAVID'
>>> my_name.lower()
'david'
\end{lstlisting}
\end{frame}

\begin{frame}[fragile]{Sting methods}
    \begin{lstlisting}[style=colorful, language=Python]
>>> my_name = "David"
>>> my_name.center(10)
'  David   '
>>> my_name.find('v')
2
>>> my_name.replace('D', 'J')
'Javid'
>>> my_name.split('v')
['Da', 'id']    
\end{lstlisting}
\end{frame}


\begin{frame}[fragile]{Mutability of Lists}
    \begin{itemize}
        \item Lists in Python are mutable, meaning their contents can be changed after creation.
        \item Example:
    \end{itemize}
    \begin{lstlisting}[style=colorful, language=Python]
my_list = [1, 3, True, 6.5]
my_list[0] = 2**10
print(my_list)  # Output: [1024, 3, True, 6.5]
    \end{lstlisting}
\end{frame}

\begin{frame}[fragile]{Immutability of Strings}
    \begin{itemize}
        \item Strings in Python are immutable, meaning their contents cannot be changed after creation.
        \item Example:
    \end{itemize}
    \begin{lstlisting}[style=colorful, language=Python]
>>> my_name = 'David'
>>> my_name[0]='X'
Traceback (most recent call last):
File "<pyshell#84>", line 1, in <module>
my_name[0]='X'
TypeError: 'str' object does not support item assignment
>>>
    \end{lstlisting}
\end{frame}

\begin{frame}{Tuples}
    \begin{itemize}
        \item Tuples are ordered collections of items
        \item Tuples can contain items of different types
        \item Tuples are immutable, meaning they cannot be changed after creation
        \item Tuples are created using parentheses \texttt{()}
    \end{itemize}
\end{frame} 

\begin{frame}[fragile]{Creating Tuples}
    \begin{lstlisting}[style=colorful, language=Python]
# Create an empty tuple
my_tuple = ()
# Create a tuple with items
my_tuple = (1, 2, 3, 4, 5)
my_another_tuple = ('a', 'b', 'c', False, 3, 0.11)
    \end{lstlisting}        
\end{frame}

\begin{frame}[fragile]
    \begin{lstlisting}[style=colorful, language=Python]
>>> my_tuple = (2,True,4.96)
>>> my_tuple
(2, True, 4.96)
>>> len(my_tuple)
3
>>> my_tuple[0]
2
>>> my_tuple * 3
(2, True, 4.96, 2, True, 4.96, 2, True, 4.96)
>>> my_tuple[0:2]
(2, True)
 >>>  
    \end{lstlisting}
\end{frame}
\begin{frame}[fragile]{Tuple Multiplication Differences}
    \begin{itemize}
      \item \textbf{Key Concept:} The comma in tuple creation is crucial.
      \item Without a comma, parentheses are treated as grouping operators.
    \end{itemize}
    \begin{lstlisting}[style=colorful, language=Python]
    >> my_tuple = (2)
    >> type(my_tuple)
    <class 'int'>
    >> my_tuple = (2,)
    >> type(my_tuple)
    <class 'tuple'>
    >> my_tuple = (1,2,3)
    >> my_tuple*2
    (1, 2, 3, 1, 2, 3)
    >> (my_tuple)*2
    (1, 2, 3, 1, 2, 3)
    >> (my_tuple,)*2
    ((1, 2, 3), (1, 2, 3))
    \end{lstlisting}
\end{frame}

\begin{frame}
    \frametitle{Sets}
    \begin{itemize}
        \item Sets are unordered collections of unique items
        \item Sets do not allow duplicate items
        \item Sets are mutable, meaning they can be changed after creation
        \item Sets are created using curly braces \texttt{\{\}}
    \end{itemize}   
\end{frame}

\begin{frame}{Set Methods in Python (Part 1)}
    \begin{table}[]
        \centering
        \begin{tabular}{|l|l|p{3.8cm}|}
            \hline
            \textbf{Method} & \textbf{Use} & \textbf{Explanation} \\ \hline
            \texttt{union} & \texttt{set1.union(set2)} & Returns a new set with all elements from both sets \\ \hline
            \texttt{intersection} & \texttt{set1.intersection(set2)} & Returns a new set with only the elements common to both sets \\ \hline
            \texttt{difference} & \texttt{set1.difference(set2)} & Returns a new set with all items from the first set not in the second \\ \hline
            \texttt{issubset} & \texttt{set1.issubset(set2)} & Asks whether all elements of one set are in the other \\ \hline
        \end{tabular}
        \caption{Set Methods in Python (Part 1)}
    \end{table}
\end{frame}

\begin{frame}{Set Methods in Python (Part 2)}
    \begin{table}[]
        \centering
        \begin{tabular}{|l|l|p{5cm}|}
            \hline
            \textbf{Method} & \textbf{Use} & \textbf{Explanation} \\ \hline
            \texttt{add} & \texttt{set.add(item)} & Adds item to the set \\ \hline
            \texttt{remove} & \texttt{set.remove(item)} & Removes item from the set \\ \hline
            \texttt{pop} & \texttt{set.pop()} & Removes an arbitrary element from the set \\ \hline
            \texttt{clear} & \texttt{set.clear()} & Removes all elements from the set \\ \hline
        \end{tabular}
        \caption{Set Methods in Python (Part 2)}
    \end{table}
\end{frame}

\begin{frame}{Mutability of Sets in Python}
    \begin{itemize}
      \item \textbf{Sets themselves are mutable.}
        \begin{itemize}
          \item Elements can be added using \texttt{add()}.
          \item Elements can be removed using \texttt{remove()} or \texttt{discard()}.
          \item Sets can be updated with \texttt{update()} or \texttt{union()}.
        \end{itemize}
      \item \textbf{Elements within a set must be immutable.}
        \begin{itemize}
          \item This means you can't directly put lists or dictionaries in a set.
          \item You can use immutable types like integers, strings, and tuples.
        \end{itemize}
      \item \textbf{Why the Confusion?}
        \begin{itemize}
          \item Some sources may loosely use "immutable" due to the element restriction.
          \item Set implementation relies on hash values, which mutable elements would disrupt.
          \item Integers, floats, strings, and tuples are hashable or immutable.
        \end{itemize}
    \end{itemize}
\end{frame}

    \begin{frame}[fragile]
        \frametitle{Creating Sets}
        \begin{lstlisting}[style=colorful, language=Python]
# Create an empty set
my_set = set()
# Create a set with items
my_set = {1, 2, 3, 4, 5}
my_another_set = {'a', 'b', 'c', False, 3, 0.11}
        \end{lstlisting}            
    \end{frame}

\begin{frame}[fragile]{Set Operations}
    \begin{lstlisting}[style=colorful, language=Python]
# Create two sets
>>> set1 = {1, 2, 3, 4, 5}  
>>> set2 = {4, 5, 6, 7, 8}
# Union of two sets
>>> union_set = set1.union(set2)
>>> print(union_set)
{1, 2, 3, 4, 5, 6, 7, 8} 
# Difference of two sets
>>> difference_set = set1.difference(set2)
>>> print(difference_set)
{1, 2, 3}
# Intersection of two sets
>>> intersection_set = set1.intersection(set2)
>>> print(intersection_set)
{4, 5}
    \end{lstlisting}
\end{frame}

\begin{frame}
    \frametitle{Dictionaries}
    \begin{itemize}
        \item Dictionaries are unordered collections of key-value pairs before Python 3.7
        \item Dictionaries are ordered collections of key-value pairs in Python 3.7 and later
        \item Dictionaries can contain items of different types
        \item Dictionaries are mutable, meaning they can be changed after creation
        \item Dictionaries are created using curly braces \texttt{\{\}}
    \end{itemize}
\end{frame}

\begin{frame}[fragile]{Creating Dictionaries}
    \begin{lstlisting}[style=colorful, language=Python]
# Create an empty dictionary
my_dict = {}
# Create a dictionary with items
my_dict = {'name': 'David', 'age': 32}
my_another_dict = {'name': 'Alice', 'age': 28, 'is_student': True}
    \end{lstlisting}
\end{frame} 

\begin{frame}
\frametitle{Dictionary Methods in Python (Part 1)}
\begin{table}[]
    \centering
    \begin{tabular}{|l|l|p{5cm}|}
        \hline
        \textbf{Method} & \textbf{Use} & \textbf{Explanation} \\ \hline
        \texttt{clear} & \texttt{a\_dict.clear()} & Removes all items from the dictionary \\ \hline
        \texttt{copy} & \texttt{a\_dict.copy()} & Returns a shallow copy of the dictionary \\ \hline
        \texttt{get} & \texttt{a\_dict.get(key,default)} & Returns the value for \texttt{key} if it exists \\ \hline
        \texttt{items} & \texttt{a\_dict.items()} & Returns key-value pairs of all items in the dictionary \\ \hline
        \texttt{keys} & \texttt{a\_dict.keys()} & Returns a view of all keys in the dictionary \\ \hline
    \end{tabular}
    \caption{Dictionary Methods in Python (Part 1)}
\end{table} 
\end{frame}

\begin{frame}
\frametitle{Dictionary Methods in Python (Part 2)}
\begin{table}[]
    \centering
    \begin{tabular}{|l|p{4.3cm}|p{4cm}|}
        \hline
        \textbf{Method} & \textbf{Use} & \textbf{Explanation} \\ \hline
        \texttt{pop} & \texttt{a\_dict.pop(key)} & Removes the item with \texttt{key} and returns its value \\ \hline
        \texttt{popitem} & \texttt{a\_dict.popitem()} & Removes and returns an arbitrary item from the dictionary \\ \hline
        \texttt{setdefault} & \texttt{a\_dict.setdefault(key, default)} & Returns the value for \texttt{key} if it exists, else sets the value to \texttt{default} \\ \hline
    \end{tabular}
    \caption{Dictionary Methods in Python (Part 2)}
\end{table}
\end{frame}


\begin{frame}
\frametitle{Dictionary Methods in Python (Part 3)}
\begin{table}[]
    \centering
    \begin{tabular}{|l|p{5cm}|p{4cm}|}
        \hline
        \textbf{Method} & \textbf{Use} & \textbf{Explanation} \\ \hline
        \texttt{update} & \texttt{a\_dict.update(other\_dict)} & Updates the dictionary with items from \texttt{other\_dict} \\ \hline
        \texttt{values} & \texttt{a\_dict.values()} & Returns a view of all values in the dictionary \\ \hline
    \end{tabular}
    \caption{Dictionary Methods in Python (Part 3)}
\end{table}
\end{frame}

\begin{frame}[fragile]{Dictionary Operations}
    \begin{lstlisting}[style=colorful, language=Python]
# Create two dictionaries
>>> dict1 = {'name': 'David', 'age': 32}
>>> dict2 = {'name': 'Alice', 'age': 28, 'is_student': True}
# Update a dictionary
>>> dict1.update(dict2)
>>> print(dict1)
{'name': 'Alice', 'age': 28, 'is_student': True}
# Get a value from a dictionary
>>> print(dict1.get('name'))
Alice
# Remove an item from a dictionary
>>> dict1.pop('age')
28
    \end{lstlisting}
\end{frame}

\begin{frame}[fragile]{Dictionary Operations}
    \begin{lstlisting}[style=colorful, language=Python]
# Create a dictionary
>>> my_dict = {'name': 'David', 'age': 32}
# Add a new item to the dictionary 
>>> my_dict['is_student'] = True
>>> print(my_dict)
{'name': 'David', 'age': 32, 'is_student': True}
# Remove an item from the dictionary
>>> del my_dict['age']
>>> print(my_dict)
{'name': 'David', 'is_student': True}
    \end{lstlisting}
\end{frame} 

\begin{frame}[fragile]{Dictionary Operations}
    \begin{lstlisting}[style=colorful, language=Python]
# Create a dictionary
>>> my_dict = {'name': 'David', 'age': 32}
# Check if a key exists in the dictionary
>>> print('name' in my_dict)
True
>>> print('is_student' in my_dict)
False
# Get the number of items in the dictionary
>>> print(len(my_dict))
2
    \end{lstlisting}
\end{frame} 


\begin{frame}[fragile]{String Comparison}
    \begin{itemize}
        \item When comparing two strings, Python compares the ``Unicode code point'' of the first character of each string.
        \item If the first characters are the same, it compares the second characters, and so on, until one of the strings ends
    \end{itemize}

    \begin{lstlisting}[style=colorful, language=Python]
>>> "10" > "2"
False
>>> ord("1")
49
>>> ord("2")
50
    \end{lstlisting}
\end{frame}

\begin{frame}
    \begin{itemize}
        \item This explains why files on your computer might be sorted in this order:
        \begin{itemize}
            \item \texttt{``Picture1.jpg''}
            \item \texttt{``Picture10.jpg''}
            \item \texttt{``Picture100.jpg''}
            \item \texttt{``Picture11.jpg''}
            \item \texttt{``Picture2.jpg''}
        \end{itemize}
    \end{itemize}
\end{frame}

