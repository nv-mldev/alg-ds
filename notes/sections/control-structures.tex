
\section{Control Structures}
\begin{frame}
    \frametitle{Control Structures}
    \begin{itemize}
        \item Control structures allow you to control the flow of your program.
        \item They include:
        \begin{itemize}
            \item Conditional statements
            \item Loops
            \item Functions
        \end{itemize}
    \end{itemize}
\end{frame}
\begin{frame}
    \frametitle{Conditional Statements}
    \begin{itemize}
        \item Conditional statements allow you to execute different blocks of code based on certain conditions.
        \item The most common conditional statement is the \texttt{if} statement.
        \item The \texttt{if} statement evaluates a condition and executes a block of code if the condition is \texttt{True}.
    \end{itemize}
\end{frame}
\subsection{If Statements}
\begin{frame}[fragile]{If Statement Syntax}
    \begin{lstlisting}[style=colorful, language=Python]
if condition:
    # code block
    \end{lstlisting}
    \begin{itemize}
        \item The \texttt{condition} is an expression that evaluates to \texttt{True} or \texttt{False}.
        \item If the \texttt{condition} is \texttt{True}, the code block is executed.
        \item If the \texttt{condition} is \texttt{False}, the code block is skipped.
    \end{itemize}
\end{frame}

\begin{frame}[fragile]{If Statement Example}
    \begin{lstlisting}[style=colorful, language=Python]
x = 10  # Assign a value to x
if x > 5:  # Check if x is greater than 5
    print("x is greater than 5")  # Print a message
    \end{lstlisting}
    \begin{itemize}
        \item In this example, the condition \texttt{x > 5} is \texttt{True} because \texttt{x} is 10.
        \item The code block is executed, and the message \texttt{"x is greater than 5"} is printed.
    \end{itemize}
\end{frame}

\begin{frame}[fragile]{If-Else Statement Syntax}
    \begin{lstlisting}[style=colorful, language=Python]
if condition:
    # code block 1
else:  
    # code block 2
    \end{lstlisting}
    \begin{itemize}
        \item If the \texttt{condition} is \texttt{True}, \texttt{code block 1} is executed.
        \item If the \texttt{condition} is \texttt{False}, \texttt{code block 2} is executed.
    \end{itemize}
\end{frame}     


\begin{frame}[fragile]{If-Else Statement Example}
    \begin{lstlisting}[style=colorful, language=Python]
x = 3  # Assign a value to x
if x > 5:  # Check if x is greater than 5
    print("x is greater than 5")  # Print a message
else:  # If x is not greater than 5
    print("x is not greater than 5")  # Print a message
    \end{lstlisting}
    \begin{itemize}
        \item In this example, the condition \texttt{x > 5} is \texttt{False} because \texttt{x} is 3.
        \item The code block after the \texttt{else} statement is executed, and the message \texttt{"x is not greater than 5"} is printed.
    \end{itemize}
\end{frame}

\begin{frame}
    \frametitle{Nested If Statements}
    \begin{itemize}
        \item You can nest \texttt{if} statements inside other \texttt{if} statements.
        \item This allows you to check multiple conditions in sequence.
        \item The inner \texttt{if} statement is only executed if the outer \texttt{if} statement's condition is \texttt{True}.
    \end{itemize}
\end{frame}

\begin{frame}[fragile]{Nested If Statements Example}
    \begin{lstlisting}[style=colorful, language=Python]
x = 10  # Assign a value to x
if x > 5:  # Check if x is greater than 5
    if x < 15:  # Check if x is less than 15
        print("x is between 5 and 15")  # Print a message
    \end{lstlisting}
    \begin{itemize}
        \item In this example, the condition \texttt{x > 5} is \texttt{True} because \texttt{x} is 10.
        \item The inner \texttt{if} statement checks if \texttt{x} is less than 15, which is \texttt{True}.
        \item The message \texttt{"x is between 5 and 15"} is printed.
    \end{itemize}
\end{frame}

\begin{frame}[fragile]{If-Elif-Else Statement Syntax}
    \begin{lstlisting}[style=colorful, language=Python]
if condition1:
    # code block 1
elif condition2:
    # code block 2
else:
    # code block 3
    \end{lstlisting}
    \begin{itemize}
        \item If \texttt{condition1} is \texttt{True}, \texttt{code block 1} is executed.
        \item If \texttt{condition1} is \texttt{False} and \texttt{condition2} is \texttt{True}, \texttt{code block 2} is executed.
        \item If both \texttt{condition1} and \texttt{condition2} are \texttt{False}, \texttt{code block 3} is executed.
    \end{itemize}
\end{frame} 


\begin{frame}
    \frametitle{Loops}
    \begin{itemize}
        \item Loops allow you to repeat a block of code multiple times.
        \item The two most common types of loops are:
        \begin{itemize}
            \item \texttt{for} loops
            \item \texttt{while} loops
        \end{itemize}
    \end{itemize}           
\end{frame}
\subsection{For Loops}
\begin{frame}[fragile]{For Loop Syntax}
    \begin{lstlisting}[style=colorful, language=Python]
for item in iterable:
    # code block
    \end{lstlisting}
    \begin{itemize}
        \item The \texttt{for} loop iterates over each \texttt{item} in the \texttt{iterable}.
        \item The \texttt{code block} is executed once for each \texttt{item} in the \texttt{iterable}.
    \end{itemize}
\end{frame}

\begin{frame}[fragile]{For Loop Example}
    \begin{lstlisting}[style=colorful, language=Python]
for i in range(5):
    print(i)
    \end{lstlisting}
    \begin{itemize}
        \item In this example, the \texttt{range(5)} function generates a sequence of numbers from 0 to 4.
        \item The \texttt{for} loop iterates over each number in the sequence and prints it.
    \end{itemize}
\end{frame}

\begin{frame}[fragile]{For Loop Example (2)}
    \begin{lstlisting}[style=colorful, language=Python]
>>> fruits = ["apple", "banana", "cherry"]
>>> for fruit in fruits:
>>>    print(fruit)
apple
banana
cherry
    \end{lstlisting}
    \begin{itemize}
        \item In this example, the \texttt{for} loop iterates over each \texttt{fruit} in the \texttt{fruits} list.
        \item The \texttt{code block} prints each \texttt{fruit} in the list.
    \end{itemize}
\end{frame}

\subsection{While Loops}
\begin{frame}[fragile]
    \frametitle{While Loop Syntax}
    \begin{lstlisting}[style=colorful, language=Python]
while condition:
    # code block
    \end{lstlisting}
    \begin{itemize}
        \item The \texttt{while} loop executes the \texttt{code block} as long as the \texttt{condition} is \texttt{True}.
        \item The \texttt{condition} is evaluated before each iteration of the loop.
    \end{itemize}
\end{frame}

\begin{frame}[fragile]{While Loop Example}
    \begin{lstlisting}[style=colorful, language=Python]
i = 0
while i < 5:
    print(i)
    i += 1
    \end{lstlisting}
    \begin{itemize}
        \item In this example, the \texttt{while} loop prints the value of \texttt{i} as long as \texttt{i} is less than 5.
        \item The value of \texttt{i} is incremented by 1 in each iteration.
    \end{itemize}
\end{frame}


\begin{frame}{List Comprehensions}
    \begin{itemize}
        \item List comprehensions are a concise way to create lists in Python.
        \item They allow you to generate a new list by applying an expression to each item in an existing list.
        \item List comprehensions are more readable and efficient than traditional \texttt{for} loops.
    \end{itemize}
\end{frame}

\begin{frame}{List Comprehension Syntax}
    \begin{itemize}
        \item The basic syntax of a list comprehension is:
        \begin{center}
            \texttt{[expression for item in iterable]}
        \end{center}
        \item The \texttt{expression} is applied to each \texttt{item} in the \texttt{iterable} to generate a new list.
    \end{itemize}
\end{frame}

\begin{frame}[fragile]{List Comprehension Example (1)}
    \begin{lstlisting}[style=colorful, language=Python]
>>> sq_list = []
>>> for x in range(1, 11):
sq_list.append(x * x)
>>> sq_list
[1, 4, 9, 16, 25, 36, 49, 64, 81, 100]
>>> sq_list = [x * x for x in range(1, 11)]
>>> sq_list
[1, 4, 9, 16, 25, 36, 49, 64, 81, 100]
    \end{lstlisting}
\end{frame}

\begin{frame}[fragile]{List Comprehension Example (2)}
    \begin{lstlisting}[style=colorful, language=Python]
>>> even_sq_list = [x * x for x in range(1, 11) if x % 2 == 0]
>>> even_sq_list
[4, 16, 36, 64, 100]
>>>[ch.upper() for ch in 'comprehension' if ch not in 'aeiou']
    \end{lstlisting}
\end{frame}

